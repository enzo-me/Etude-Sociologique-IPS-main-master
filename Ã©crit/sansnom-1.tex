\documentclass[a4paper, 11pt]{article}

\usepackage{lmodern} % Police standard sous LaTeX : Latin Modern
\usepackage[french]{babel} % Pour la langue française
\usepackage[utf8]{inputenc} % Pour l'UTF-8
\usepackage[T1]{fontenc} % Pour les césures des caractères accentués
\usepackage{amsmath,amsfonts, amssymb}
\usepackage{geometry}
\geometry{margin=2cm}
\let\oldforall\forall
\renewcommand{\forall}{\oldforall\,}
\everymath{\displaystyle} 
\usepackage{titlesec}

\titleformat{\section}{\normalfont\Large\bfseries}{\thesection}{1em}{}
\titleformat{\subsection}{\normalfont\large\bfseries}{\thesubsection}{1em}{}
\titleformat{\subsubsection}{\normalfont\normalsize\bfseries}{\thesubsubsection}{1em}{}


\begin{document}

\title{Lien entre IPS et goûts d'élèves de}
\author{
   \bsc{Enzo Merioud}
   \and
   \bsc{Rayan Moustafid Dos Santos}
   \and
    \bsc{Monsieur S. Gibaud}\thanks{Directeur d'étude}
} 
\date{Jeudi 11 Janvier 2024} 

\maketitle

\tableofcontents

\newpage

\section{Introduction}
\subsection{Motivations}

\subsection{Prérequis Mathématiques, la régression linéaire}

Pour chaque individu \(i\), la variable expliquée s'écrit comme une fonction linéaire des variables explicatives :
\begin{align}
y_i &= \beta_0 + \beta_1x_{i,1} + \cdots + \beta_Kx_{i,k} + \varepsilon_i
\end{align}

Où :

- \(y_i\) est la variable dépendante (la variable que vous essayez de prédire) pour l'individu \(i\).
- \(\beta_0\) est l'intercept, la valeur de \(y\) lorsque toutes les variables explicatives sont égales à zéro.
- \(\beta_1, \beta_2, \ldots, \beta_K\) sont les coefficients de régression, représentant la relation entre la variable dépendante \(y\) et les variables explicatives \(x_1, x_2, \ldots, x_K\).
- \(x_{i,1}, x_{i,2}, \ldots, x_{i,K}\) sont les valeurs des variables explicatives correspondantes pour l'individu \(i\).
- \(\varepsilon_i\) est le résidu, l'erreur ou la différence entre la valeur observée \(y_i\) et la valeur prédite \(\hat{y}_i\) (la valeur obtenue à partir du modèle).

Pour trouver les valeurs des \(\beta_0, \beta_1, \beta_2, \ldots, \beta_K\) et \(\varepsilon_i\), vous utilisez la méthode des moindres carrés ordinaires (MCO) qui vise à minimiser la somme des carrés des résidus. Cette méthode consiste à ajuster le modèle pour qu'il soit aussi proche que possible des données observées.

Les résidus, \(\varepsilon_i\), sont les différences entre les valeurs observées de \(y_i\) et les valeurs prédites par le modèle (\(y_i - \hat{y}_i\)). Ils mesurent l'erreur de prédiction du modèle pour chaque individu de l'échantillon.

\subsection{Organisation de l'étude et résultats obtenus}
\subsubsection{Résultats Attendus}

Le But de l'étude était de trouver un lien entre l'IPS des élève et leurs goûts en HGGSP. Effectivement, nous espérions que le coefficients de l'IPS soit le plus important. Malheureusement, d'après la régression linéaire, ce qui influence le plus les goûts des élèves c'est ce que préfère enseigner les professeurs.

\subsubsection{Résultats Obtenus}

Nous avons donc chercher d'autres résultats avec les données que nous possédons, et c'est ce que l'on a trouvé.\\
Nous avons d'abord chercher un lien entre \textbf{sexe} des professeurs et \textbf{IPS}. Les coefficients étaient particulièrement élevés mais négatifs ($-3.02801885^{14}$ et $-3.02801885^{14}$).\\
Nous avons ensuite chercher un lien entre \textbf{âge} des professeurs et \textbf{IPS}. Les coefficients sont cette fois positifs et suffisament hauts. Plus l'IPS est haut plus l'âge des professeur l'est aussi. Ce résultat est particulièrement intéressant, effectivement, cela montre que les professeurs plus âgés ont tendance à allés dans des lycées d'élèves plus aisés.\\
Finalement, nous avons fait une régression linéaire. Mais cette fois-ci, l'IPS est la donnée expliquée. Le coefficient le plus haut parmis les données explicatives est celui des professeurs qui ont fait une Classe préparatoire ECS ou équivalent. Tandis que le plus faible e st celui des professeurs qui ont suivis une Faculté de géographie – niveau licence ou équivalent. Ces résultats sont particulièrement intéressants puisque cela montre que lorsque le niveau d'étude des professeurs est haut, l'IPS aussi. Tandis que lorsque le niveau d'étude des professeurs est plutôt bas, l'IPS baisse. Cela peut aller à l'inverse de la pensée commune. Effectivment, on aurait tendance à penser que les professeurs ayant un plus fort niveau d'étude sont placés dans les lycées les moins favorisés.

\subsubsection{Approximations faites}
Par manque d'informations, des approximations ont du être fait durant l'étude. Effectivement, certains professeur n'avait renseigné que le nom de leur académie. Dans ce cas là, nous mettons l'IPS moyen en voie GT de l'académie.

\newpage

\section{Corps de l'étude}


\end{document}